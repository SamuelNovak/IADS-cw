\documentclass{article}
\usepackage[utf8]{inputenc}
\usepackage{listings}
\usepackage{amsmath}
\usepackage{wasysym}
\usepackage{amssymb}
\usepackage{hyperref}
\usepackage{xcolor}
\usepackage{tabularx}

\newcommand{\todo}[1]{\textcolor{red}{$\star$~TODO: \textbf{#1}}}
\newcommand{\R}{\mathbb{R}}
\newcommand{\randint}{\operatorname{RANDINT}}

\newcommand{\bb}[1]{\textbf{#1}}
\newcommand{\uu}[1]{\underline{#1}}
\newcommand{\ii}[1]{\textit{#1}}

\newenvironment{pyc}{\begin{lstlisting}[language=Python]}{\end{lstlisting}}
\newcommand{\py}[1]{\lstinline[language=Python]{#1}}

\renewcommand{\thesection}{\Alph{section}}
\setcounter{section}{2}


\title{IADS Coursework 3}
\author{Samo Nov\'ak (s1865783)}
\date{March 2020}

\begin{document}

\maketitle

\section{Algorithm: Nearest Neighbour Insertion}

\todo{Name? Call it \bb{expansion}?}

We draw inspiration for this algorithm from the outline of Nearest Insertion
(and other TSP heuristics) by Nilsson\cite{nilsson}. We chose this particular
algorithm because of its elegance - it follows a very simple principle:
start with some trivial circuit of minimum weight and then the most suitable
nodes to add to it, until all nodes are used.


\section{Experiments}

In this part, we were to generate (or write a generator for) many different
test cases and quantitatively measure how each heuristic (Swap, 2-Opt,
Greedy, \todo{own}) performs.

\subsection{Test generation}

Firstly, let us summarize the assumptions and setting of the problem:
\begin{itemize}
\item We have an \uu{undirected}, \uu{complete} graph $G=(V,E)$ where $|V|=n$.
\item We have a weight function $w:E\to\R^{+}$, such that
  $\forall i,j\in V \ldotp w(i,j) = w(j,i)$, which will be represented as
  a \uu{symmetric} matrix \py{Graph.dists[i][j]}\,$= w(i,j)$).
  We assume $\forall i\in V\ldotp w(i,i)=0$.
\item We have three cases of our problem: \uu{general} TSP
  (no other constraints), \uu{metric} TSP ($w$ is a metric, so
  $\forall u,v,x\in V\ldotp w(u,v)\le w(u,x) + w(x,v)$), and
  \uu{Euclidean} TSP (the graph is embedded in a plane, this is also metric).
\end{itemize}

The test cases could be generated as completely random graphs. For small graphs
(\todo{how big?}), it is then feasible to compute the best path, so that we can
compare it to the results from the tested heuristics. However, for larger graphs,
it will be more useful to generate a graph with some pre-planned best path.

\subsubsection{Small random graphs}

Generating these will be simple: we shall populate the matrix
\py{Graph.dists} with random positive integers. Then we will
compute the best TSP path.

In the \uu{Euclidean} case, we will instead generate randomly positioned
cities in a plane.

For a general \uu{metric} case, we will generate the cities randomly embedded
in a plane, same as in the Euclidean case. However, we will use
the \uu{taxicab} metric (also called the \uu{Manhattan} metric), which measures
the distance along the axes individually and sums them. In the case of discrete
positions (which we will use):
\begin{align*}
  & w_M: \mathbb{N}^n \times \mathbb{N}^n \to \mathbb{N} \\
  & w_M\left(\left<u_1,...,u_n\right>, \left<v_1,...,v_n\right>\right) = \sum_{i=1}^n \left|u_i - v_i\right|
\end{align*}
where $n$ is the number of dimensions of the space (in our case $n=2$).

Since all these graphs will be small, we can get the optimal TSP solution by
simply trying out all the permutations.

\subsubsection{Larger graphs}

A good way to intentionally generate an optimal path in a graph is to create
a cycle with small weights in it and then assign larger values to all the other
edges. So:
\begin{align*}\forall i\in\{0,...,n-2\}\ldotp w(i,i+1) & =\randint(1,5)\\
  \forall i,j\in{0,...,n-1}\ldotp i\ne j \wedge j\ne i+1 \ldotp
  w(i,j) & = \randint(6,10) \\
\end{align*}
The random number boundaries here are somewhat arbitrary, we just need to make
sure that the cycle has smaller values.

Once we have this matrix, we swap the vertices around, so that the initial
permutation $[0,1,...,n-1]$ won't immediately be the solution.


\bibliographystyle{abbrv}
\begin{thebibliography}{9}
\bibitem{nilsson}
  Christian Nilsson,
  \textit{Heuristics for the Traveling Salesman Problem},
  Link\"oping University,
  2003.
  Accessible online [27-03-2020]: \url{https://pdfs.semanticscholar.org/7b80/bfc1c5dd4e10ec807c6f56d0f31f8bf86bc6.pdf}
\end{thebibliography}

\end{document}
